\documentclass[10pt]{beamer}

\usepackage[utf8]{inputenc}
% Remove russian in English presentation
\usepackage[english,russian]{babel}

\usetheme[
  sectionpage=simple,
  numbering=fraction
]{metropolis}
\usepackage{appendixnumberbeamer}

\usepackage{booktabs}
\usepackage[scale=2]{ccicons}

\usepackage{pgfplots}
\usepgfplotslibrary{dateplot}

\usepackage{xspace}
\newcommand{\themename}{\textbf{\textsc{metropolis}}\xspace}

\usepackage{todonotes}
\usepackage{ifthen}%
\providecommand\enabletodos{true}%
\ifthenelse{ \equal{\enabletodos}{true} }{%
  \presetkeys{todonotes}{inline}{}%
}{%
  \presetkeys{todonotes}{disable}{}%
}%

\setbeamertemplate{caption}{\raggedright\insertcaption\par}

\AtBeginSection[]{}

\setbeamertemplate{section in toc}{%
  \alert{$\bullet$}~\inserttocsection}
\setbeamercolor{subsection in toc}{bg=white,fg=structure}
\setbeamertemplate{subsection in toc}{%
  \hspace{1.2em}{\alert{\rule[0.3ex]{3pt}{3pt}}~\inserttocsubsection\par}}

\usepackage{hyperref}
\hypersetup{unicode=true}

\usepackage{adjustbox}
\protected\def\psverb#1{\def\innerpsverb##1#1{\texttt{##1}}\innerpsverb}
\usepackage{listings}

\usepackage{makecell}
\usepackage{changepage}

\title{Автоматическая генерация сигнатур сетевых протоколов и приложений}
%\subtitle{Подназвание доклада}
\date{18 апреля 2024 г.}
\author{Алексей Дурнов}

% Russian
\institute{МФТИ, кафедра системного программирования, ИСП РАН}
\titlegraphic{\hfill\includegraphics[height=0.5cm]{logo_isp_ru.png}}
% English
% \institute{ISP RAS}
% \titlegraphic{\hfill\includegraphics[height=0.5cm]{logo_isp_en.png}}

\definecolor{Blue}{HTML}{29619b}
\setbeamercolor{frametitle}{bg=Blue}
\setbeamercolor{palette primary}{bg=Blue}

\begin{document}

\maketitle

\begin{frame}{Введение}
    \begin{itemize}
      \item Методы классификации сетевого трафика:
        \begin{enumerate}
            \item основанные на идентификации по номеру порта
            \item основанные на DPI подходе
            \begin{itemize}
                \item \alert{сигнатурный}
            \end{itemize}
            \item основанные на статистических характеристиках
        \end{enumerate}
        \item Свойства сигнатур для сетевых протоколов и приложений:
        \begin{enumerate}
            \item короткие общие подстроки
            \item несколько потоков с разным набором подстрок
            \item необходимость частого обновления
        \end{enumerate}
    \end{itemize}
\end{frame}

\begin{frame}{Цель}
    Целью данной работы является разработка и реализация метода автоматической
    генерации сигнатур полезной нагрузки сетевого трафика для классификации этого трафика
    в соответствии с использующимся протоколом или приложением в режиме реального времени.
\end{frame}

\begin{frame}{Задачи}
    \begin{itemize}
        \item Провести исследование литературы по соответствующей теме.
        \item Собрать набор сетевых трасс для последующего тестирования и сравнения методов.
        \item Разработать формат хранения сигнатуры.
        \item Разработать алгоритм генерации сигнатур.
        \begin{enumerate}
            \item Выбрать методы для автоматической генерации сигнатур.
            \item Найти ограничения рассматриваемых методов.
            \item Выбрать оптимальный набор параметров метода.
        \end{enumerate}
        \item Разработать классификатор сетевого трафика для проверки сгенерированных сигнатур.
        \begin{enumerate}
            \item Выбрать и реализовать алгоритм сопоставления сигнатур
            \item Выбрать метрики, по котором можно оценить качество классификации сигнатур
        \end{enumerate}
        \item Встроить генератор сигнатур и классификатор как модули в систему анализа высокоскоростного сетевого трафика, разрабатываемую в ИСП РАН.
    \end{itemize}
\end{frame}

\appendix

\begin{frame}[standout] \vfill Спасибо за внимание \vfill \end{frame}

\end{document}
