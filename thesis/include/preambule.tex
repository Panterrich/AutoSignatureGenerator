%%% Работа с русским языком
\usepackage{cmap}			 % поиск в PDF
\usepackage{mathtext} 		 % русские буквы в формулах
\usepackage[T2A]{fontenc}	 % кодировка
\usepackage[utf8]{inputenc}	 % кодировка исходного текста
\usepackage[russian]{babel}	 % локализация и переносы

%%% Пакеты для работы с математикой
\usepackage{amsmath,amsfonts,amssymb,amsthm,mathtools}
\usepackage{icomma}

%% Номера формул
%\mathtoolsset{showonlyrefs=true} % Показывать номера только у тех формул, на которые есть \eqref{} в тексте.
%\usepackage{leqno}               % Немуреация формул слева

%% Шрифты
\usepackage{euscript}	 % Шрифт Евклид
\usepackage{mathrsfs}    % Красивый матшрифт

%% Поля (геометрия страницы)
\usepackage[left=3cm,right=1.5cm,top=2cm,bottom=2cm,bindingoffset=0cm]{geometry}

%% Русские списки
\usepackage{enumitem}
\makeatletter
\AddEnumerateCounter{\asbuk}{\russian@alph}{щ}
\makeatother

\usepackage{tcolorbox}
\usepackage{url}
\usepackage{svg}

\usepackage[ddmmyyyy,hhmmss]{datetime}
\renewcommand{\dateseparator}{.}

%%% Работа с картинками
\usepackage[nooneline]{caption}
\captionsetup[table]{justification=raggedright,labelsep=endash}
\captionsetup[figure]{justification=centering,labelsep=endash,name=Рисунок}

\usepackage{graphicx}                  % Для вставки рисунков
\graphicspath{{images/}{images2/}{../images/}{../data/}}     % папки с картинками
\setlength\fboxsep{3pt}                % Отступ рамки \fbox{} от рисунка
\setlength\fboxrule{1pt}               % Толщина линий рамки \fbox{}
\usepackage{wrapfig}                   % Обтекание рисунков и таблиц текстом

%%% Работа с таблицами
\usepackage{array,tabularx,tabulary,booktabs} % Дополнительная работа с таблицами
\usepackage{longtable}                        % Длинные таблицы
\usepackage{multirow}                         % Слияние строк в таблице
\usepackage{setspace}

%% Красная строка
\setlength{\parindent}{12.5mm}
\usepackage{indentfirst}

%% Интервалы
\onehalfspacing

%% TikZ
\usepackage{tikz}
\usetikzlibrary{graphs,graphs.standard}

%% Верхний колонтитул
\usepackage{fancyhdr}
\pagestyle{fancy}

%% Перенос знаков в формулах (по Львовскому)
\newcommand*{\hm}[1]{#1\nobreak\discretionary{}{\hbox{$\mathsurround=0pt #1$}}{}}

%% дополнения
\usepackage{float}   % Добавляет возможность работы с командой [H] которая улучшает расположение на странице
\usepackage{gensymb} % Красивые градусы
\usepackage{caption} % Пакет для подписей к рисункам, в частности, для работы caption*
\usepackage{listings} % Пакет для листингов с кодом
\usepackage{xcolor}
\usepackage[noadjust]{cite}

\definecolor{codegreen}{rgb}{0,0.6,0}
\definecolor{codegray}{rgb}{0.5,0.5,0.5}
\definecolor{codepurple}{rgb}{0.58,0,0.82}
\definecolor{backcolour}{rgb}{0.95,0.95,0.92}

\lstdefinestyle{mystyle}{
    backgroundcolor=\color{backcolour},
    commentstyle=\color{codegreen},
    keywordstyle=\color{magenta},
    numberstyle=\tiny\color{codegray},
    stringstyle=\color{codepurple},
    basicstyle=\ttfamily\small,
    breakatwhitespace=true,
    columns=flexible,
    breaklines=true,
    captionpos=t,
    keepspaces=true,
    numbers=left,
    numbersep=5pt,
    showspaces=false,
    showstringspaces=false,
    showtabs=false,
    tabsize=4
}

\lstset{style=mystyle}

% подключаем hyperref (для ссылок внутри  pdf)
\usepackage[unicode, pdftex]{hyperref}
