\begin{abstract}

    \begin{center}
    \large{Автоматическая генерация сигнатур сетевых протоколов} \\[0.5 cm]
    \large{\textit{Дурнов Алексей Николаевич}} \\[1 cm]
    \end{center}

    Данная работа посвящена исследованию и разработке методов автоматической генерации сигнатур сетевых протоколов.
    В ходе работы был проведен анализ существующих подходов к генерации сигнатур.
    На их основе был выбран наиболее универсальный формат сигнатур для сетевого трафика.
    Были изучены ограничения представленных методов генерации сигнатур и выбран для последующих исследований
    метод LASER (Application Signature ExtRaction),
    основанный на алгоритме LCS (Longest Common Subsequence).
    По их результатам было установлено, что сборка TCP-сессии, частичная или полная,
    негативно сказывается на результатах классификации для данного метода:
    полученные сигнатуры являются слишком специфичными и не обладают достаточной предсказательной способностью.
    Наилучшие результаты классификации показал стандартный метод LASER без сборки TCP-сессии: средневзвешенное значение F1-меры составило $0,95$.
    Также было показано, что недостаточно простой постобработки для выделения минимального покрывающего набора сигнатур.
    Разработанный генератор сигнатур и классификатор были встроены как модули в систему анализа сетевого трафика, разрабатываемую в ИСП РАН.

\end{abstract}
\newpage
