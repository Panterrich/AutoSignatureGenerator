\section{Заключение}
\label{sec:Section5} \index{Section5}

В данной работе был разработан и реализован автоматический генератор сигнатур на основе полезной нагрузки сетевого трафика
и классификатор трафика, основанный на сопоставлении этих сигнатур, в соответствии с использующимся протоколом.

Были выполнены следующие задачи:
\begin{enumerate}
    \item Проведено исследование литературы по соответствующей теме.
    \item Собран набор сетевых трасс для генерации и классификации.
    \item Выбран формат сигнатуры сетевых протоколов.
    \item Рассмотрены ограничения выбранных методов и реализован один из них.
    \item Разработан классификатор сетевого трафика для проверки сгенерированных сигнатур.
    \item Рассмотрено влияние сборки TCP-сессии и постобработки на результат классификации.
    \item Встроены генератор сигнатур и классификатор как модули в систему анализа высокоскоростного сетевого трафика, разрабатываемую в ИСП РАН.
\end{enumerate}

Планируемые будущие исследования по данной теме:

\begin{enumerate}
    \item Реализация и сравнение других методов генерации сигнатур: AutoSig и SigBox.
    \item Реализация модуля уточнения положения сигнатуры и его влияние на точность.
    \item Рассмотрение возможности применения машинного обучения для отбора сигнатур при постобработке результатов.
\end{enumerate}

\newpage
