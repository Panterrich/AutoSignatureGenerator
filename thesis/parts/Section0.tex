\section{Введение}
\label{sec:Section0} \index{Section0}

Интернет-провайдеры и сетевые администраторы хотят идентифицировать тип сетевого трафика, который проходит через их сеть, для того,
чтобы предоставлять своим клиентам лучший сервис, предлагая им высокое качество обслуживание (QoS),
а также планировать свою инфраструктуру и управлять ею. Сетевой трафик можно классифицировать как в соответствии с использующимся протоколом,
так и в соответствии с использующимся приложением. Вторая классификация более трудоёмкая, но позволяет решать более широкий спектр задач:
формирование трафика в сети, улучшение качества обслуживание, а также предоставление более детального биллинга.

В области классификации сетевого трафика было проведено множество исследований, что привело к разработке многих методов.
Самый наивный метод классификации сетевого трафика это идентификация по номеру порта.
На заре развития интернета многие клиент-серверные приложения имели четко определенные номера портов сервера,
которые определяются Internet Assigned Numbers Authority (IANA) \cite{IANA}.
К примеру, DNS использует номер порта сервера 53,
SMTP использует номер порта сервера 25, HTTP - номер 80 и т.д.
Таким образом, используя известные номера портов, можно классифицировать трафик.
Однако современные приложения используют динамическое распределение портов и туннелирование трафика, например, по протоколу HTTP,
поэтому этот метод на данный момент не обладает необходимой точностью \cite{dusi2009tunnel}.
Чтобы преодолеть эти ограничения идентификации были введены более совершенные методы \cite{getman2015analys}.

В рамках подхода глубокого анализа пакетов (DPI) анализатор просматривает содержимое каждого пакета полностью.
Эта технология получило очень широкое распространение,
так как эти методы являются точными для идентификации трафика, несмотря на свою трудо- и ресурсоёмкость.
Ведущий подход, использующийся в DPI для классификации трафика, основан на сопоставлении сигнатур полезной нагрузки.
Под сигнатурой понимается часть данных полезной нагрузки,
которая является статичной и различимой для приложений/протоколов и может быть описана, как последовательность строк или шестнадцатеричных чисел.

Подход, альтернативный DPI, основан на статистическом анализе.
Статистический анализ реализуются в основном с помощью алгоритмов машинного обучения \cite{erman2006qrp05}.
Для этого подхода используются уже косвенные признаки пакетов и потоков, такие как задержки между пакетами, размеры пакетов и другие.
Существует несколько проблем при классификации сетевого трафика с использованием данного подхода.
Во-первых, статистические характеристики, используемые при классификации, нестабильны, например, задержка и
коэффициент потери пакетов в сети динамичны.
Во-вторых, не все потоки конкретного приложения имеют особые характеристики трафика.
Потоки, принадлежащие разным приложениям, могут иметь схожую статистику. Трудно отличить эти похожие потоки с помощью их свойств.
Несмотря на эти проблемы, этот подход может применяться для зашифрованного сетевого трафика.
Так как в первом приближении можно считать, что полезная нагрузка зашифрованного трафика представляет собой белый шум,
поэтому сигнатурный подход не применим.

Чтобы точно классифицировать поток, система DPI должна искать сигнатуры в полезной нагрузке пакетов.
Такой подход не нов, системы обнаружения вторжения (IDS) с помощью сигнатур находят интернет-червей и другой трафик, угрожающий безопасности
\cite{singh2004automated, kim2004autograph, newsome2005polygraph}.
По сравнению с проблемой генерации сигнатур червей проблема генерации сигнатур безопасного трафика имеет несколько отличий.
Во-первых, общие подстроки в прикладных протоколах обычно очень короткие, а некоторые составляют всего несколько байт,
в то время как общие подстроки у червей обычно достигают несколько десятков или сотен байт.
Некоторые методы \cite{singh2004automated, kim2004autograph} оказываются неэффективными, когда общие подстроки короткие.
Во-вторых, разные потоки в одном и том же приложении могут иметь разные общие подстроки.
Например, в P2P приложении некоторые потоки используются для обмена одноранговой информацией,
а другие потоки используются для обмена данными. Они могут использовать разные протоколы и иметь разные общие подстроки.

Поначалу сигнатуры извлекались вручную, но и сейчас часто пишутся в том числе на заказ \cite{amonitoring}.
Постоянное появление новых приложений и их частые обновления подчёркивают необходимость автоматической генерации сигнатур,
так как ручная операция извлечения сигнатур занимает много времени, а также может быть разница в качестве сигнатур
в зависимости от оператора извлечения.

Методы автоматической генерации сигнатур должны быть основаны не на семантическом анализе протоколов, так как,
хотя они и повышают точность сигнатур, но не могут быть применены
к анализу высокоскоростного трафика в режиме реального времени \cite{park2008towards}.

Поэтому данная работа посвящена исследованию различных методов автоматической генерации сигнатур полезной нагрузки
для классификации сетевых протоколов и сложностям, связанным с классификацией приложений.

\newpage
