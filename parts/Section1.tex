\section{Постановка задачи}
\label{sec:Section1} \index{Section1}

Целью данной работы является разработка и реализация метода автоматической генерации сигнатур полезной нагрузки сетевого трафика
для классификации этого трафика в соответствии с использующимся протоколом или приложением в режиме реального времени.

Для достижения поставленной цели необходимо решить следующие задачи:

\begin{enumerate}
    \item Провести исследование литературы по соответствующей теме.
    \item Собрать набор сетевых трасс для последующего тестирования и сравнения методов.
    \item Разработать формат хранения сигнатуры.
    \item Разработать алгоритм генерации сигнатур.
    \begin{enumerate}
        \item Выбрать методы для автоматической генерации сигнатур.
        \item Найти ограничения рассматриваемых методов.
        \item Выбрать оптимальный набор параметров метода для каждого тестируемого протокола и приложения,
        если метод обладает настраиваемыми параметрами.
    \end{enumerate}
    \item Разработать классификатор сетевого трафика для проверки сгенерированных сигнатур.
    \begin{enumerate}
        \item Выбрать и реализовать алгоритм сопоставления сигнатур
        \item Выбрать метрики, по котором можно оценить качество классификации сигнатур
    \end{enumerate}
    \item Встроить генератор сигнатур и классификатор как модули в систему анализа высокоскоростного сетевого трафика, разрабатываемую в ИСП РАН.
\end{enumerate}

\newpage
