\section{Введение}
\label{sec:Chapter0} \index{Chapter0}

Интернет-провайдеры и сетевые администраторы хотят идентифицировать тип сетевого трафика, который проходит через их сеть, для того,
чтобы предоставлять своим клиентам лучший сервис, предлагая им высокое качество обслуживание (QoS),
а также планировать свою инфраструктуру и управлять ею. Сетевой трафик можно классифицировать как в соответствии с использующимся протоколом,
так и в соответствии с использующимся приложением. Вторая классификация более трудоёмкая, но позволяет решать более широкий спектр задач:
формирование трафика в сети, улучшение качества обслуживание, а также предоставление более детального биллинга.

В области классификации сетевого трафика было проведено множество исследований, что привело к разработке многих методов.
Самый наивный метод классификации сетевого трафика это идентификация по номеру порта.
Однако современные приложения используют динамическое распределение портов и туннелирования трафика, например, по протоколу HTTP,
данный метод даёт очень плохие и неточные результаты. Чтобы преодолеть эти ограничения идентификации были введены более совершенные методы.

Первый подход основан на сопоставлении сигнатур полезной нагрузки. Сигнатура полезной нагрузки - это часть данных полезной нагрузки,
которая является статичной и различимой для приложений и может быть описана, как последовательность строк или шестнадцатиричных чисел.
Второй подход основан на алгоритмах машинного обучения. Для этого подхода используются такие признаки потоков и пакетов,
как задержки между пакетами, размеры пакетов и другие, а полезная нагрузка пакетов не анализируется, поэтому он менее точен, чем сигнатурный подход.
Однако он может применяться для зашифрованного сетевого трафика, так как в приближении полезная нагрузка зашифрованного трафика представляет собой белый шум,
поэтому сигнатурный подход невозможен. Также существуют и гибридные методы, которые используют эти два подхода вместе.

Методы, которые анализируют полезную нагрузку пакетов, являются очень эффективными и точными для идентификации трафика.
Их часто называют глубокой проверкой пакетов (DPI). DPI является очень трудоемким и ресурсоемким процессом.
Система DPI должна искать сигнатуры в полезной нагрузке пакетов, чтобы точно классифицировать поток. Такой подход не нов, системы обнаружения вторжения (IDS),
основанные на DPI, с помощью сигнатур находят интернет-червей и другой трафик, угрожающий безопасности.
Далее рассматриваемые методы будут сосредаточены на идентификации приложений среди безопасностного трафика.

Поначалу сигнатуры извлекались вручную. Постоянное появление новых приложений и их частые обновления подчёркивают необходимость автоматической генерации сигнатур,
так как ручная операция извлечения сигнатур занимает много времени, а также может быть разница в качестве сигнатур в зависимости от оператора извлечения.
Методы автоматической генерации сигнатур должны быть основаны не на семантическом анализе протоколов, так как хотя они и повышают точность сигнатур,
но не могут быть применены к анализу высокоскоростного трафика в режиме реального времени.

Поэтому данная работа посвещена исследованию различных методов автоматической генерации сигнатур полезной нагрузки для классификации сетевого трафика по протоколам и приложением.

\newpage
